% !TEX TS-program = pdflatex
% !TEX encoding = UTF-8 Unicode

% This is a simple template for a LaTeX document using the "article" class.
% See "book", "report", "letter" for other types of document.

\documentclass[11pt]{article} % use larger type; default would be 10pt

\usepackage[utf8]{inputenc} % set input encoding (not needed with XeLaTeX)

%%% Examples of Article customizations
% These packages are optional, depending whether you want the features they provide.
% See the LaTeX Companion or other references for full information.

%%% PAGE DIMENSIONS
\usepackage{geometry} % to change the page dimensions
\geometry{a4paper} % or letterpaper (US) or a5paper or....
% \geometry{margin=2in} % for example, change the margins to 2 inches all round
% \geometry{landscape} % set up the page for landscape
%   read geometry.pdf for detailed page layout information

\usepackage{graphicx} % support the \includegraphics command and options

% \usepackage[parfill]{parskip} % Activate to begin paragraphs with an empty line rather than an indent

%%% PACKAGES
\usepackage{verbatim} % adds environment for commenting out blocks of text & for better verbatim
\usepackage{subfig} % make it possible to include more than one captioned figure/table in a single float
% These packages are all incorporated in the memoir class to one degree or another...
\usepackage{pdfpages}
\usepackage{minted}
\usepackage{geometry}
\geometry{
  top=0.5in,            % <-- you want to adjust this
  inner=0.5in,
  outer=0.5in,
  bottom=0.5in,
  headheight=3ex,       % <-- and this
  headsep=2ex,          % <-- and this
}
\usepackage{mdframed}
\renewcommand{\abstractname}{\vspace{-\baselineskip}}

%%% HEADERS & FOOTERS
\usepackage{fancyhdr} % This should be set AFTER setting up the page geometry
\pagestyle{fancy} % options: empty , plain , fancy
\renewcommand{\headrulewidth}{1pt} % customise the layout...
\lhead{jps111@uclive.ac.nz}\chead{Jesse Sheehan}\rhead{53366509}
\lfoot{}\cfoot{\thepage}\rfoot{}

%%% SECTION TITLE APPEARANCE
%\usepackage{sectsty}
%\allsectionsfont{\sffamily\mdseries\upshape} % (See the fntguide.pdf for font help)
% (This matches ConTeXt defaults)

%%% END Article customizations

\newcommand{\includecode}[2][c]{\begin{mdframed}\inputminted[linenos=true, breaklines]{#1}{#2}\end{mdframed}}

\title{COSC 264 Assignment}
\date{\today}
\author{Jesse Sheehan (Student ID: 53366509)}

\begin{document}

\begin{titlepage}

\maketitle

\hrulefill

\begin{abstract}
I decided to use C in this assignment so I could practice it for ENCE260. Also I think it's a bit more fun to program in C.

Towards the end of the assignment I became aware of C's struct and enum language features. If I were to redo this project I would use structs to represent the DT-Request and DT-Response packets and enums to represent things like language codes, packet request type, etc.

I have been very liberal in the use of comments and have used Javadoc-style comments for providing more information about functions. This, unfortunately, makes the source code quite a long read (table \ref{tab:1}).

\begin{table}[H]
	\centering
  \begin{tabular}{ |l|r|r|r|r| }
	\hline
	Language & Files & Blank & Comment & Code \\ \hline
	C & 5 & 240 & 387 & 775 \\
	C Header & 4 & 22 & 10 & 60 \\ \hline \hline
	Total & 9 & 262 & 397 & 835 \\ \hline
  \end{tabular}
  \caption{A lines-of-code breakdown of the C source.}
  \label{tab:1}
\end{table}

\end{abstract}

\tableofcontents

\end{titlepage}
\setcounter{page}{2}
\includepdf[addtotoc={1,section,1,Plagiarism Declaration,sec:plag}]{includes/PlagiarismDeclaration.pdf}

\section{Source Code Listings}

\subsection{Utilities}
Contains several general helper functions.
\includecode{../src/utils.h}
\includecode{../src/utils.c}

\newpage
\subsection{Protocol}
Contains functions and definitions that are relevant to both the client and the server, and more specifically, the protocol.
\includecode{../src/protocol.h}
\includecode{../src/protocol.c}

\newpage
\subsection{Server}
Contains functions that pertain only to the server.
\includecode{../src/server.h}
\includecode{../src/server.c}

\newpage
\subsection{Client}
Contains functions that pertain only to the client.
\includecode{../src/client.h}
\includecode{../src/client.c}

\newpage
\subsection{Protocol Testing}
Contains functions that test the integrity of the protocol functions.
\includecode{../src/test/protocol.test.c}

\newpage
\subsection{Makefile}
Used to build the aforementioned source listings.
\includecode[make]{../Makefile}

\end{document}
